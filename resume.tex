% !TEX program = xelatex

\documentclass{resume}
\usepackage{hyperref}
\usepackage{xcolor}
%\usepackage{zh_CN-Adobefonts_external} % Simplified Chinese Support using external fonts (./fonts/zh_CN-Adobe/)
%\usepackage{zh_CN-Adobefonts_internal} % Simplified Chinese Support using system fonts

\begin{document}
\pagenumbering{gobble} % suppress displaying page number

\name{Qingcheng Zhao}

\basicInfo{
  \email{zhaoqch1@shanghaitech.edu.cn} \textperiodcentered\
  \phone{(+86) 153-875-01026} \textperiodcentered\
  \github[Github]{https://github.com/Clarivy} \textperiodcentered\
  \homepage[blog]{http://clarivy.github.io/}
}

\section{\faGraduationCap\ Education}
\datedsubsection{\textbf{ShanghaiTech University}, Shanghai, China}{Sep. 2021 -- Jun. 2025(Expected)}
\textit{Bachelor of Engineering} in Computer Science.\par
\textit{Advisors: \href{http://www.yu-jingyi.com/}{Prof. Jingyi Yu} and \href{https://www.xu-lan.com/}{Prof. Lan Xu}}\par
\textit{Overall GPA 3.69/4, ranked 31/178}

\datedsubsection{\textbf{University of California Berkeley}, California, United States of America}{Aug. 2023 -- Dec. 2023}
\textit{GLOBE Program} in College of Engineering, University Of California Berkeley \par

\textit{Overall GPA 3.67/4}

\section{\faSearch\ Research Interests}


My research interests lie at the intersection of \textbf{3D Vision}, \textbf{Generative AI}. I am particularly interested in developing generative models for high-fidelity \textbf{3D scene reconstruction, rendering, and interaction}, with applications in virtual environments, creative content creation, and embodied AI systems. My work explores both the creation of \textbf{3D representations} and the integration of \textbf{human-centric priors} to enable context-aware and emotionally responsive interactions in 3D environments. I aim to advance the capabilities of generative AI by bridging the gap between perception and synthesis for real-world and immersive applications.

\section{\faBook\ Publications}

\begin{itemize}
  \item {\small \cite{zhao2024media2face} Media2Face: Co-speech Facial Animation Generation With Multi-Modality Guidance} \hfill{\textbf{SIGGRAPH 2024}}\\
        \textit{\footnotesize \textbf{Qingcheng Zhao*}, Pengyu Long*, Qixuan Zhang, Dafei Qin, Han Liang, Longwen Zhang, Yingliang Zhang, Jingyi Yu, Lan Xu}\\
        \textit{(\href{https://sites.google.com/view/media2face}{Project Page})}
        \textit{(\href{https://dl.acm.org/doi/10.1145/3641519.3657413}{Paper})}
      
  \item {Single-view Panoptic Reconstruction with Instance-level Diffusion Priors} \hfill{\textbf{Under Review}}\\
        \textit{\small \textbf{Qingcheng Zhao}, Xiang Zhang, Zeyuan Chen, Yuan Gao, Zhuowen Tu}\\

  \item {Zero-Shot Single Image Panoptic Reconstruction} \hfill{\textbf{Under Review}}\\
        \textit{\small Yuan Gao, Xiang Zhang, Zeyuan Chen, \textbf{Qingcheng Zhao}, Zhuowen Tu}\\
\end{itemize}


\section{\faFlask\ Research Experience}

\datedsubsection{\textbf{ShanghaiTech University}}{ Mar. 2022 - Present}
\role{Research Assistant. Advised by \href{https://scholar.google.com/citations?user=R9L_AfQAAAAJ}{Prof. Jingyi Yu} and \href{https://scholar.google.de/citations?user=aPS5pJkAAAAJ}{Prof. Lan Xu}}

\datedline{\textbf{\small Co-speech Facial Animation Generation With Multi-Modality Guidance}}{SIGGRAPH 2024}

\begin{itemize}
  \item Proposed a diffusion model in latent motion space for co-speech facial animation generation, accepting rich multi-modality guidance.
  \item Built an efficient variational auto-encoder mapping facial geometry and images to a highly generalized and decoupled expression latent space for expressions and identities.
  \item Achieved state-of-the-art performance on multiple datasets, outperforming existing methods in terms of both quality and diversity.
\end{itemize}

\datedsubsection{\textbf{University of California San Diego}}{ Jul. 2024 - Present}
\role{Visiting Scholar. Advised by \href{https://pages.ucsd.edu/~ztu/}{Prof. Zhuowen Tu}}

\datedline{\textbf{\small Single View 3D Scene Reconstruction With Generative Prior}}{In Progress}
\begin{itemize}
  \item Proposed a diffusion model for panoptic 3D scene reconstruction from a single RGB image.
  \item Presented a novel generative approach using a tri-plane 2D unet diffusion model conditioned on a projected 3D prior to reconstruct 3D scenes with an efficient yet effective latent space.
\end{itemize}

\section{\faUsers\ Industry Experience}

\datedsubsection{\textbf{Nvidia Corporation} Shanghai, China}{Feb. 2024 -- Jun. 2024}
\role{Software Development Engineer (Internship)}

\begin{itemize}
  \item Built a LLM-powered agent for gameplay with human-like behaviors in most popular games, using a text-based game UI descriptor to interact with GPT-4.
  \item Enhanced the language model with a generalizable visual understanding module to improve the agent's performance in various games.
  \item Widely deployed in production environment, reduced the human labor and time cost significantly, enabling full automation of the game testing process with minimal configuration.
\end{itemize}

\datedsubsection{\textbf{Deemos Technologies Inc. } Shanghai, China}{Nov. 2022 -- Feb. 2024}
\role{Intern Researcher}

\begin{itemize}
  \item Built a real-time 3D interactive avatar system utilizing audio-driven facial expression animation technologies at Global AI developer Conference 2023.
  \item Built a web application for \href{https://hyperhuman.deemos.com/}{ChatAvatar} project based on DreamFace\cite{zhang2023dreamface}, which can generate 3D avatars from a single image or text prompt.
\end{itemize}

% \datedsubsection{\textbf{DJI Technology Co., Ltd.} Shanghai, China}{Jun. 2022 -- Nov. 2022}
% \role{Software Development Engineer (Internship)}

% Built real-time visualization based on Qt utilizing ADB to collect, analyze, and visualize data from drones. Optimized the existing system for better performance and reduced processing time for data analysis.

% \cite{zaharia2012resilient}
% Reference Test
%\datedsubsection{\textbf{Paper Title\cite{zaharia2012resilient}}}{May. 2015}
%An xxx optimized for xxx\cite{verma2015large}
%\begin{itemize}
%  \item main contribution
%\end{itemize}

\section{\faInfo\ Activities}

\datedsubsection{\textbf{GeekPie Association}}{Sep. 2021 - Present}
\role{President of GeekPie Association}

\begin{itemize}
  \item Developed \homepage[CourseBench]{https://coursebench.geekpie.club/}, a highly regarded web application that allow students to comment and discuss about courses.
  \item Organize tech workshop and provide technical support in algorithms for members and new students.
\end{itemize}

\datedsubsection{\textbf{CS100: Introduction to Programming}}{2023, 2024}
\role{Senior Teaching Assistant}

\begin{itemize}
  \item Give office hours and \href{https://github.com/GKxxQAQ/CS100-recitations-spring2023}{recitation classes};
  \item Assist with homework assignments and corrections.
  \item Won the SIST Outstanding Teaching Assistant Award.
\end{itemize}



\section{\faTrophy\ Honors and Awards}
\datedline{\textit{\textbf{ShanghaiTech International Exchange Program First-Class Scholarship}}, ShanghaiTech University}{2024}
\datedline{\textit{\textbf{Outstanding Teaching Assistant}}, SIST, ShanghaiTech University}{2023}
\datedline{\textit{\textbf{Merit Student}(top 2\%)}, ShanghaiTech University}{2022}
\datedline{\textit{\textbf{Outstanding Officer}}, ShanghaiTech Student Union}{2022}
\datedline{\textit{\textbf{Bronze Medal}}, Award on The 2021 China Collegiate Programming Contest, Harbin Site }{2021}
\datedline{\textit{\textbf{One-hundred Fourth Place}}, Award on The 2021 ICPC Asia-East Continent Final Contest}{2021}


\section{\faCogs\ Skills}
\begin{itemize}[parsep=0.5ex]
  \item Programming Languages: Python > C/C++ >= Javascript == Typescript > Ruby > MATLAB
  \item Tools: PyTorch, Blender, OpenCV, Git, \LaTeX, Docker, Vue, React, FastAPI, Node.js, Rancher, Kubernetes,
\end{itemize}

%% Reference
%\newpage
\bibliographystyle{IEEETran}
\bibliography{mycite}
\end{document}
