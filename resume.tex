% !TEX program = xelatex

\documentclass{resume}
\usepackage{hyperref}
\usepackage{xcolor}
%\usepackage{zh_CN-Adobefonts_external} % Simplified Chinese Support using external fonts (./fonts/zh_CN-Adobe/)
%\usepackage{zh_CN-Adobefonts_internal} % Simplified Chinese Support using system fonts

\begin{document}
\pagenumbering{gobble} % suppress displaying page number

\name{Qingcheng Zhao}

\basicInfo{
  \email{zhaoqch1@shanghaitech.edu.cn} \textperiodcentered\
  \phone{(+86) 153-875-01026} \textperiodcentered\
  \github[Github]{https://github.com/Clarivy} \textperiodcentered\
  \homepage[blog]{http://clarivy.github.io/}
}

\section{\faGraduationCap\ Education}
\datedsubsection{\textbf{ShanghaiTech University}, Shanghai, China}{2021 -- Present}
\textit{Bachelor of Engineering} in Computer Science. Expected graduation date: June 2025\par
\textit{Advisors: \href{http://www.yu-jingyi.com/}{Prof. Jingyi Yu} and \href{https://www.xu-lan.com/}{Prof. Lan Xu}}\par
\textit{Overall GPA 3.7/4, ranked 31/178}

\datedsubsection{\textbf{University of California Berkeley}, California, United States of America}{Aug. 2023 -- Dec. 2023}
\textit{GLOBE Program} in College of Engineering, University Of California Berkeley \par

\textit{Overall GPA 3.67/4}

\section{\faSearch\ Research Inserests}

My primary research interest lies in \textbf{3D Vision}, with a specific focus on achieving high-fidelity \textbf{3D reconstruction, rendering and driving} of digital humans. Concurrently, I am exploring \textbf{generative modeling} techniques to generating, editing and interacting with common objects, and large-scale scenes.

\section{\faBook\ Publications}

\begin{itemize}
  \item {\small \href{https://dl.acm.org/doi/10.1145/3641519.3657413}{Media2Face}: Co-speech Facial Animation Generation With Multi-Modality Guidance} \hfill{\textbf{SIGGRAPH 2024}}\\
  \textit{\footnotesize \textbf{Qingcheng Zhao*}, Pengyu Long*, Qixuan Zhang, Dafei Qin, Han Liang, Longwen Zhang, Yingliang Zhang, Jingyi Yu, Lan Xu}\\
  \textit{(\href{https://sites.google.com/view/media2face}{\color{blue}{Project Page}})}
  \textit{(\href{https://dl.acm.org/doi/10.1145/3641519.3657413}{\color{blue}{Paper}})}
\end{itemize}



\section{\faUsers\ Experience}

\datedsubsection{\textbf{ShanghaiTech University}}{ Mar. 2022 - Present}
\role{Undergraduate Researcher. Advised by \href{http://www.yu-jingyi.com/}{Prof. Jingyi Yu} and \href{https://www.xu-lan.com/}{Prof. Lan Xu}}

First author and presenter of \href{https://dl.acm.org/doi/10.1145/3641519.3657413}{Media2Face} at \href{https://s2024.conference-program.org/presentation/?id=papers_294&sess=sess116}{\textbf{SIGGRAPH 2024}}.
Media2Face can generate highly realistic and expressive 3D facial animations from diverse multimedia inputs—audio, text, and images, trained on the largest ever co-speech 3D facial animation dataset.

\datedsubsection{\textbf{University of California San Diego}}{ Jul. 2024 - Present}
\role{Visiting Scholar. Advised by \href{https://pages.ucsd.edu/~ztu/}{Prof. Zhuowen Tu}}

Developing a Transformer-Diffusion model for panoptic 3D scene reconstruction from a single RGB image. 


% \datedsubsection{\textbf{Nvidia Corporation} Shanghai, China}{Feb. 2024 -- Jun. 2024}
% \role{Software Development Engineer (Internship)}

% Build a LLM-based agent for gameplay with human-like behaviors in most popular games, enabling full automation of the game testing process with minimal configuration. Widely deployed in the production environment. The agent is trained on a large-scale synthetic data.

\datedsubsection{\textbf{Deemos Technologies Inc. } Shanghai, China}{Nov. 2022 -- Nov. 2023}
\role{Intern Researcher}

\begin{itemize}
  \item Build a real-time 3D interactive avatar system utilizing audio-driven facial expression animation technologies at Global AI developer Conference 2023.
  \item Build a web application for \href{https://hyperhuman.deemos.com/}{ChatAvatar} project based on \href{https://dl.acm.org/doi/abs/10.1145/3592094}{DreamFace}, which can generate 3D avatars from a single image or text prompt.
\end{itemize}

\datedsubsection{\textbf{DJI Technology Co., Ltd.} Shanghai, China}{Jun. 2022 -- Nov. 2022}
\role{Software Development Engineer (Internship)}

Built real-time visualization based on Qt utilizing ADB to collect, analyze, and visualize data from drones. Optimized the existing system for better performance and reduced processing time for data analysis.

% Reference Test
%\datedsubsection{\textbf{Paper Title\cite{zaharia2012resilient}}}{May. 2015}
%An xxx optimized for xxx\cite{verma2015large}
%\begin{itemize}
%  \item main contribution
%\end{itemize}

\section{\faTrophy\ Honors and Awards}
\datedline{\textit{\textbf{Special Scholarship} for Undergraduate Overseas Exchange Program}, ShanghaiTech University}{2024}
\datedline{\textit{\textbf{Merit Student}}, ShanghaiTech University}{2022}
\datedline{\textit{\textbf{Bronze Medal}}, Award on The 2021 China Collegiate Programming Contest, Harbin Site }{2021}
\datedline{\textit{\textbf{One-hundred Fourth Place}}, Award on The 2021 ICPC Asia-East Continent Final Contest}{2021}

\section{\faInfo\ Activities}

\datedsubsection{\textbf{GeekPie Association}}{Sep. 2021 - Present}
\role{President of GeekPie Association}

\begin{itemize}
  \item Developed a highly regarded web application that simplifies the course selection process for all students at ShanghaiTech University. People can comment and discuss freely at our platform. Check it out at \homepage[coursebench]{https://coursebench.geekpie.club/}.
  \item Designed and developed the homepage for the Frontier of Life Science and Technology ShanghaiTech University (FoLST2021) conference, helping to increase awareness and participation in the event. Check out the source code at \homepage[github]{https://github.com/Clarivy/FoLST2021}.
\end{itemize}

\datedsubsection{\textbf{CS100: Introduction to Programming}}{2023, 2024}
\role{Senior Teaching Assistant}

\begin{itemize}
  \item Designed assignments for student engagement.
  \item Led bi-weekly \href{https://github.com/GKxxQAQ/CS100-recitations-spring2023}{recitation classes}.
  \item Collaborated with Professor \href{https://www.xu-lan.com/}{Lan Xu} on grading and exam supervision.
  \item Recipient of the SIST Outstanding Teaching Assistant Award.
\end{itemize}



\section{\faCogs\ Skills}
\begin{itemize}[parsep=0.5ex]
  \item Programming Languages: Python > C++ >= Javascript == Typescript > Ruby > MATLAB > C
  \item Tools: PyTorch, Blender, OpenCV, Git, Vim, VSCode, Jupyter, \LaTeX, Markdown, Docker, Vue, React, FastAPI, Uvicorn, Node.js, Gradio, Rancher, Kubernetes, 
\end{itemize}

%% Reference
%\newpage
%\bibliographystyle{IEEETran}
%\bibliography{mycite}
\end{document}
